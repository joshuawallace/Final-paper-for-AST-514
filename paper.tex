\documentclass{emulateapj}
\usepackage{graphicx}
\usepackage{natbib}
\usepackage{amssymb}
\usepackage{amsmath}
\usepackage{float}
%\usepackage{float}

\def\prd{Phys. Rev. D}

\def\jcap{J. Cosmology Astropart. Phys.}
                % Journal of Cosmology and Astroparticle Physics

\def\bminus{$\beta^-$}
\def\Msol{M\ensuremath{_{\odot}}}


\begin{document}

\title{Uncertainty in r-process and s-process Sites}

\author{Joshua Wallace}
\affil{AST 514, Department of Astrophysical Sciences, Princeton
  University, Princeton, NJ 08544}

\begin{abstract}
Both the r- and s-process are important for
producing nuclides heavier than $^{56}$Fe.  A brief history of the
development of the r- and s-process is presented as well as the
relevant nuclear physics.  The focus of this work
is on the details of the r- and s-processes themselves, especially the
proposed sites where they are operative. Details of the processes
themselves as well as certain observations give important clues as to
what these sites might be.
  The s-process abundances consist of
2-3 components.  The main s-process
component is thought to happen during thermal flashes in low mass AGB
stars, the weak s-process component is thought to happen during core
helium burning, and the strong s-process component can be explained by
having the main s-process component operate in stars with low
metallicity.  The nature of the r-process strongly suggests that it
operates in transient phenomena.  Strongest candidates for the site of the r-process consist of proto-neutron star
winds and binary neutron star mergers.  These proposed sites are
investigated.
\end{abstract}

\keywords{nuclear reactions, nucleosynthesis, abundances}


\section{Introduction}
The creation of the variety of elements and isotopes observed both terrestrially and throughout the universe is a topic not just of astrophysical importance but also important for understanding our origin as living organisms.  The relationship between our existence as living creatures and nucleosynthesis is colloquially summed up as ``we are made from stardust''.  While this is technically true (Big Bang nucleosynthesis aside), extant and developing research in nucleosynthesis show that the full story of how the elements and isotopes came to be is far more complicated than a five word quip.  There are a variety of proposed sites and processes that all contribute to the measured elemental and isotopic abundances of the Galaxy which, so far, still do not form a completely consistent picture.  This makes astrophysical nucleosynthesis still an active topic of research.

The field of nucleosynthesis was born...(history lesson)

\section{History of Development of Nucleosynthesis}
The field of astrophysical nucleosynthesis can be regarded as starting with the proposal of \cite{eddington1920} that stars have some sort of internal energy source and that the fusion of hydrogen into helium could be that energy source.  Subsequent developments in quantum mechanics revealed how such fusion is possible via quantum tunneling (e.g., \citealt{gurneycondon1928}; \citealt{atkinson1929}). Later, \cite{hoyle1946} and \cite{hoyle1954} discussed the production of elements heavier than helium through fusion inside stars, with $^{56}$Fe being the heaviest element produced.

Figure {\bf include figure} shows the binding energy per nucleon as a function of increasing atomic number.  Binding energy per nucleon is minimized for the atomic species $^{56}$Ni (which decays into $^{56}$Fe).  Because of this the fusion of elements heavier than $^{56}$Fe is endothermic rather than exothermic and stars cannot maintain such reactions before collapsing.  The nucleosynthesis mechanisms that produce elements heavier than $^{56}$Fe were first explained and predicted in \cite{burbidge1957}.  {\bf Possibly include their figure on page 16?}).  Among these mechanisms and the focus of this paper are the slow neutron capture process, or s-process, and the rapid neutron capture process, or r-process.  
Since then there has been some debate about the locations where these processes are effective, which will be investigated in Sections~\ref{sec:s} and~\ref{sec:r}.

\section{The Physics Behind Nucleosynthesis}
The majority of the nuclei produced since the Big Bang have been
produced through processes that involve stars {\bf citation?}.  One
notable exception to this is cosmic rays, which produce elements such
as Li, Be, and B (\citealt{meneguzzi1971}) and isotopes such as
$^{14}$C.  The classical picture of stellar nucleosynthesis is of main
sequence stars
building H into $^4$He in their cores then, in subsequent stages of
evolution, increasingly heavier elements up to and including
$^{56}$Fe. This pathway has been immortalized in song and popular
culture but it only accounts for a small fraction of all the elements
that exist in the universe and only one of multiple isotopes for each
element that stars are able to produce in their cores.  To fully
understand the chemical evolution of stars, galaxy, and the universe
we must understand the other processes involved in nucleosynthesis.

I first introduce some relevant physics.  Figure {\bf Figure number}
shows what is referred to as the ``Chart of the Nuclides,'' (hereafter
CON) which
shows all isotopes of the elements as well as whether they are
stable.  The vertical axis represents the atomic number $Z$ while the
horizontal axis represents the neutron number $N$.  Species of the
same $A=Z+N$ are positioned to the upper left and lower right from
each other.  Nuclear reactions can be represented by specific steps in
the CON.  Most relevant to this work are the neutron
capture and \bminus\ decay reactions.  Neutron capture on a given
species $X$ is given by
\begin{equation}
\label{eq:nc}
^{A}_ZX + n \rightarrow ^{A+1}_{\ \ \ Z}X'
\end{equation}
and is represented in the CON by a step from species $X$ to the
species directly to the right of it.  Similarly, \bminus\ decay of a
given species $X$ is given by
\begin{equation}
\label{eq:bd}
^{A}_ZX  \rightarrow ^{\ \ \ A}_{Z-1}X' + e^-
\end{equation}
and is represented in the CON by a step from species $X$ to the
species immediately up and the left of it.  For the r- and s-processes
these are the most important steps in the CON.

The r- and s-process are both involve neutron capture.  This is
significant because neutrons have a relatively short half-life; thus
the presence of free neutrons, required for neutron capture, depends
on there being a ready source of neutrons.  Additionally, the free
neutron density is highly dependent on the details of the neutron
source, because decaying neutrons need to be replenished to maintain a
given density.

\section{s-process}

\subsection{Description}
The s-process ({\it s} means {\it slow neutron capture}) occurs when
the rate for neutron capture is much lower than the rate
for $bminus$ decay.  Since rates for $bminus$ decay vary greatly
it is difficult to define a specific $bminus$ decay half-life as ``the
reference'' for specifying an upper-limit to the s-process neutron
capture rate.  Indeed, even in typical
s-process trajectories there are radioactive isotopes with
sufficiently long half-lives to disrupt a canonical s-process
trajectory (which will be defined later) but, in practice, such
disruptions are just small variations on an overall theme.  

Since the decay rate of radioactive isotopes is higher than the
neutron capture rate s-process trajectories through the chart of the
nuclides remain very close to the valley of stability.


\subsection{Proposed Sites}


\section{r-process}

\label{sec:r}
\subsection{Description}

The r-process occurs when neutron
densities and energies are high enough that the timescale for
neutron capture is much shorter than for $\beta^-$ decay.  Nuclear
evolution in this case does not follow the valley of stability but
instead passes on the neutron-rich side of the valley of stability.
Eventually as the neutron capture rate decreases the nuclei will have
an opportunity to \bminus\ decay back to the valley of stability.

One piece of physics very relevant to this process is the so-called
``neutron drip line.''  A nucleus with a given $Z$ can only have so
many neutrons added before it starts to become energetically favorable
for neutrons to leave the nucleus.  When this number of neutrons is
reached the addition of another neutron causes a
neutron to promptly be emitted (or ``dripped'') from the nucleus.
This maximum number of neutrons a nucleus can hold expressed as a
function of $Z$ describes the neutron drip line and it is the hard
edge seen on the neutron-rich side of the CON.

Because the r-process does not follow the valley of stability like the
s-process does, it is impossible to describe a canonical r-process
trajectory as was done for the s-process.  Instead a general picture
will be described illustrating the general principles that determine
each of a multitude of r-process trajectories.  The r-process begins 
with heavy seed nuclei, such as $^{56}$Fe.  The
rapid succession of neutron captures on these nuclei cause a
signficant neutron enrichment of the nuclei; the results of this
neutron enrichment depend on many factors such as neutron density,
temperature, and the decay times of the created nuclides, but
generally one of several things can happen while the neutron source is
active: neutron capture proceeds
faster than any \bminus\ decays, producing neutron-rich species that
run up against the neutron drip line; \bminus\ decays can occur at a
rate comparable to the neutron capture rate and nuclei can move up in
$Z$ and $A$ on a neutron-rich path that roughly parallels the valley 
of stability; nuclei capture neutrons until their neutron number 
runs up against a nuclear magic number; the low neutron capture
cross-section can effectively shut down further neutron enrichment of
the nuclei; or a nucleus with a magic number of neutrons captures a
neutron and moves to another region of the CON where any of the above
scenarios can happen again.

After the neutron source is turned off the neutron-rich nuclei are
able to \bminus\ decay to the valley of stability.  Evidence that
something like the r-process is at work in the universe can be seen by
the abundance peaks in {\bf Figure r-process peak offset}.  As
mentioned in Section~\ref{sec:s} 
the lower neutron capture cross-section of nuclides with a magic
number of neutrons there is a pile up of the abundances of these
nuclides.   Because the r-process moves on the neutron-rich side
the valley of stability the nuclei with a magic number of neutrons
have less protons than the corresponding nuclei of the s-process;
thus, in this case the r-processed nuclei will be at a lower $A$ than the
s-processed nuclei both before and after \bminus\ decaying to the
valley of stability and the corresponding abundance peak of the
r-processed nuclei will be at lower $A$ than the s-process peak.  {\bf
Figure shows this}.  Additionally, because the number of neutron magic
number nuclei accessible to the r-process is greater than the
corresponding nuclei for the s-process, the r-process peaks are wider
than the s-process peaks.  The existence of these peaks is one of the
strongest evidences for the existence of something like the r-process.

Another strong evidence for something like the r-process being in
operation in the universe is the natural existence of Th and Ur.
These elements cannot be created by the s-process because of the
s-process termination described by Equations~\ref{eq:firsttermination}
and~\ref{eq:secondtermination}; their creation requires a process that
can bridge over nuclides with short decay times.

As was mentioned in Section~\ref{sec:s} there are some stable nuclides
that are inaccessible by the s-process due to ``weak links'' in the
reaction chains that would otherwise contribute to their formation,
the weak links being nuclides with fast decay rates that \bminus\
decay back to the valley of stability; the formation of these nuclides
is completely attributable to the r-process.  Similarly, there are
some nuclides that are inaccessible by the r-process and whose
formation can be completely attributed to the s-process.  {\bf Figure}
shows such a nuclide.  r-process reation chains would end up at ?? and
thus would be unable to access any isobars of ?? with higher $Z$.  ??
is shielded from the r-process by ?? and thus cannot be formed by the
s-process.


\subsection{Proposed Sites}


\section{Conclusion}
Although the existence of the r- and s-process has been recognized for
over 50 years the exact sites where these processes happen are still
not certain. Much of theory and observation point to a few sites: low
mass AGB stars and high mass He core burning stars for the s-process,
proto-neutron stars or merging neutron stars for the r-process;
however lack of full understanding of the relevant nuclear and stellar
physics as well as direct observational evidence for all but low
mass AGB stars keep us from knowing for certain.  Increasingly
powerful computer simulations will allow us to constrain things more
fully on the theory side.  Should the Galaxy be obliging in the next
few years, direct nearby observations of either a core collapse
supernova or neutron star merger can also provide direct evidence for
the r-process site.  Whether or not we are that lucky, the field of
nucleosynthesis will likely continue to be a fertile and productive
field of research in the years to come.



\bibliographystyle{apj}
\bibliography{bibliography}

\end{document}

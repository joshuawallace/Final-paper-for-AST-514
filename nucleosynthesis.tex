The majority of the nuclei produced since the Big Bang have been
produced through processes that involve stars {\bf citation?}.  One
notable exception to this is cosmic rays, which produce elements such
as Li, Be, and B (\citealt{meneguzzi1971}) and isotopes such as
$^{14}$C.  The classical picture of stellar nucleosynthesis is of main
sequence stars
building H into $^4$He in their cores then, in subsequent stages of
evolution, increasingly heavier elements up to and including
$^{56}$Fe. This pathway has been immortalized in song and popular
culture but it only accounts for a small fraction of all the elements
that exist in the universe and only one of multiple isotopes for each
element that stars are able to produce in their cores.  To fully
understand the chemical evolution of stars, galaxy, and the universe
we must understand the other processes involved in nucleosynthesis.

I first introduce some relevant physics.  Figure {\bf Figure number}
shows what is referred to as the ``Chart of the Nuclides,'' (hereafter
CON) which
shows all isotopes of the elements as well as whether they are
stable.  The vertical axis represents the atomic number $Z$ while the
horizontal axis represents the neutron number $N$.  Species of the
same $A=Z+N$ are positioned to the upper left and lower right from
each other.  Nuclear reactions can be represented by specific steps in
the CON.  Most relevant to this work are the neutron
capture and \bminus\ decay reactions.  Neutron capture on a given
species $X$ is given by
\begin{equation}
\label{eq:nc}
^{A}_ZX + n \rightarrow ^{A+1}_{\ \ \ Z}X' + \gamma
\end{equation}
and is represented in the CON by a step from species $X$ to the
species directly to the right of it.  Similarly, \bminus\ decay of a
given species $X$ is given by
\begin{equation}
\label{eq:bd}
^{A}_ZX  \rightarrow ^{\ \ \ A}_{Z+1}X' + e^- + \overline{\nu}_e
\end{equation}
and is represented in the CON by a step from species $X$ to the
species immediately up and the left of it.  For the r- and s-processes
these are the most important steps in the CON.

The r- and s-process are both involve neutron capture.  This is
significant because neutrons have a relatively short half-life; thus
the presence of free neutrons, required for neutron capture, depends
on there being a ready source of neutrons.  Additionally, the free
neutron density is highly dependent on the details of the neutron
source, because decaying neutrons need to be replenished to maintain a
given density.

Also important to the r- and s-process are the nuclear magic numbers.
Nuclides with neutron numbers equal to one of these magic numbers are
more stable than other nuclides close to them in ($Z$,$A$) space.
These magic numbers include 2, 8, 20, 28, 50, 82, and 126, with the
latter three numbers being most relevant to the examination of this
paper.  Nuclides with these magic numbers of neutrons also have 
smaller neutron capture cross-sections, which affects the
relative abundances of nuclides produced by these processes.  The
specifics will be discussed in Sections~\ref{sec:s} and~\ref{sec:r}.
It should be noted that protons also have the same magic numbers;
nuclei with a magic number of both protons and neutrons are called
``doubly magic'' and are even more stable than singly magic nuclei.  

It should also be mentioned that many of the stable proton-rich
nuclides are formed via proton capture (known as the p-process).  {\bf
Figure} shows an example of such a nuclide.  This paper will not focus
on any discussion of p-processed nuclides and such nuclides will not
figure in my discussions.

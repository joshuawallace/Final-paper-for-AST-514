The majority of the nuclei produced since the Big Bang have been
produced through processes that involve stars {\bf citation?}.  One
notable exception to this is cosmic rays, which produce elements such
as Li, Be, and B (\citealt{meneguzzi1971}) and isotopes such as
$^{14}$C.  The classical picture of stellar nucleosynthesis is of main
sequence stars
building H into $^4$He in their cores then, in subsequent stages of
evolution, increasingly heavier elements up to and including
$^{56}$Fe. This pathway has been immortalized in song and popular
culture but it only accounts for a small fraction of all the elements
that exist in the universe and only one of multiple isotopes for each
element that stars are able to produce in their cores.  To fully
understand the chemical evolution of stars, galaxy, and the universe
we must understand the other processes involved in nucleosynthesis.

\subsection{The Physics of Nucleosynthesis}
I first introduce some relevant physics.  Figure {\bf Figure number}
shows what is referred to as the ``Chart of the Nuclides,'' (hereafter
CON) which
shows all isotopes of the elements as well as whether they are
stable.  The vertical axis represents the atomic number $Z$ while the
horizontal axis represents the neutron number $N$.  Species of the
same $A=Z+N$ are positioned to the upper left and lower right from
each other.  Nuclear reactions can be represented by specific steps in
the CON.  Most relevant to this work are the neutron
capture and \bminus\ decay reactions.  Neutron capture on a given
species $X$ is given by
\begin{equation}
^{A}_ZX + n \rightarrow ^{A+1}_ZX'
\end{equation}
and is represented in the CON by a step from species $X$ to the
species directly to the right of it.  Similarly, \bminus\ decay of a
given species $X$ is given by
\begin{equation}
^{A}_ZX  \rightarrow ^{A}_{Z-1}X' + \beta^-
\end{equation}
and is represented in the CON by a step from species $X$ to the
species immediately up and the left of it.  For the r- and s-processes
which we will consider here these are the most important steps in the CON.

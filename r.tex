\label{sec:r}
\subsection{Description}

The r-process occurs when neutron
densities and energies are high enough that the timescale for
neutron capture is much shorter than for $\beta^-$ decay.  Nuclear
evolution in this case does not follow the valley of stability but
instead passes on the neutron-rich side of the valley of stability.
Eventually as the neutron capture rate decreases the nuclei will have
an opportunity to \bminus\ decay back to the valley of stability.

One piece of physics very relevant to this process is the so-called
``neutron drip line.''  A nucleus with a given $Z$ can only have so
many neutrons added before it starts to become energetically favorable
for neutrons to leave the nucleus.  When this number of neutrons is
reached the addition of another neutron to the nucleus causes a
neutron to promptly be emitted (or ``dripped'') from the nucleus.
This maximum number of neutrons a nucleus can hold expressed as a
function of $Z$ describes the neutron drip line and it is the hard
edge seen on the neutron-rich side of the chart of the nuclides.


\subsection{Proposed Sites}
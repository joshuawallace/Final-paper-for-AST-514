\label{sec:r}
\subsection{Description}

The r-process occurs when neutron
densities and energies are high enough that the timescale for
neutron capture is much shorter than for $\beta^-$ decay.  Nuclear
evolution in this case does not follow the valley of stability but
instead passes on the neutron-rich side of the valley of stability.
Eventually as the neutron capture rate decreases the nuclei will have
an opportunity to \bminus\ decay back to the valley of stability.

Several pieces of relevant physics must be introduced.  One, nuclear
magic numbers, was already covered in Section~\ref{sec:nuc}.  The
lower neutron capture cross-sections of nuclei with magic numbers of
neutrons has a significant impact on the flow of nuclei toward
increasing $A$ in the r process.  Another important piece 
 is the so-called
``neutron drip line.''  A nucleus with a given $Z$ has a maximum
number of neutrons it can keep bound to it; any additional neutrons
that are captured once this number is reached causes a neutron to be
emitted (or ``dripped'') from the nucleus.  So neutron capture at 
the neutron drip line permits no change in $A$ or $Z$.  
This maximum number of neutrons a nucleus can hold expressed as a
function of $Z$ describes the neutron drip line and it is the hard
edge seen on the neutron-rich side of the CON.  Also relevant is
photodisintegration.  At the temperatures expected for the r-process
photons have enough energy to liberate neutrons from the nucleus:
$^A_Z$X$+\gamma \rightarrow _{\ \ \ Z}^{A-1}$X$+n$.  This process
works against neutron capture.

A qualitative description of the r-process follows.  Because the 
r-process does not follow the valley of stability like the
s-process does, it is more difficult to describe a canonical r-process
trajectory than for the s-process.  Instead a general picture
will be described illustrating the general principles that determine
each of a multitude of r-process trajectories.  A
rapid succession of neutron captures on seed nuclei causes a
signficant neutron enrichment of the nuclei.  As this proceeds there
are several things that can happen.  Additional neutron captures can
further increase the $A$ of the nuclei.  A \bminus\ decay even can
happen. Photodisintegration can
remove neutrons from the nucleus.  Nuclei can run up against the
neutron drip line, where they remain until a \bminus\ decay or
photodisintegration event move them off the drip line.  Nuclei can
get a magic number of neutrons, where lower neutron capture
cross-sections keep them from progressing as quickly to the next $A$.
These highlight some of the important individual events that can
happen in the r-process. After the neutron source is turned off the 
neutron-rich nuclei are
able to \bminus\ decay to the valley of stability.  
%Because of this
%beeline to the valley of stability after a neutron source is turned
%off, there is only one r-process nuclide 


A simple r-process model for $T\geq 1$ GK and constant $N_n \geq 10^{21}$
cm$^{-3}$ is given in \cite{iliadis2008}.  For temperatures and
neutron densities this high, both neutron capture and
photodisintegration occure much faster than \bminus\ decays.  In this
case the abundance evolution of a species $^A_Z$X is given by
\begin{multline}
\frac{N(Z,A+1)}{N(Z,A)} = - N_nN(Z,A) \langle \sigma v \rangle_{Z,A} \\
+N(Z,A+1)\times\lambda_\gamma (Z,A+1)
\end{multline}
where $N(Z,A)$ is the number density of  nuclide $^A_Z$X, $\langle 
\sigma v \rangle_{Z,A}$ is the neutron
capture reaction rate per particle pair for the same nuclide, and
$\lambda_\gamma (Z,A+1)$ is the photodisintegration constant for
nuclide  $^{A+1}_{\ \ \ Z}$X. 
In the case that the forward and reverse
reactions of neutron capture and photodisintegration reach equilibrium
the abundance ratios for two adjacent isotopes can be expressed by the
Saha equation
\begin{multline}
\label{eq:saha}
\frac{N(Z,A+1)}{N(Z,A)} =N_n\left(\frac{h^2}{2\pi
(Am_{n})kT}\right)^{3/2} \times \\
\frac{2j_{Z,A+1}+1}{(2j_{Z,A}+1)(2j_{n}+1)}
\frac{G^{\textrm{norm}}_{Z,A+1}}{G^{\textrm{norm}}_{Z,A}}e^{Q_{n\gamma}/kT}
\end{multline}
where $j_{Z,A} (j_n)$ is the angular momentum quantum number for
$^{A}_{Z}$X ($n$, a single neutron), $G^{\textrm{norm}}_{Z,A}$ is a normalized parition
function (in what follows both $j_{Z,A}$ and
$G^{\textrm{norm}}_{Z,A}$ are largely ignored and so no more specific
definitions are given), and $Q_{n\gamma}$ is the Q-value for the forward 
neutron capture
reaction on $^{A}_{Z}$X (equivalently the neutron separation energy of
$^{A+1}_{\ \ \ Z}$X). Equation~\ref{eq:saha} can be used 
for a specific $T$ and $N_n$ to find what value of 
$Q_{n\gamma}$  gives $N(Z,A+1) \approx N(Z,A)$.
For example, \cite{iliadis2008} calculates that with $T=1.25$ GK and 
$N_n=10^{22}$ cm$^{-3}$ and neglecting spin and partition function terms
$N(Z,A+1) \approx N(Z,A)$ for $Q_{n\gamma}\approx$ 3 MeV.  Generally,
$Q_{n\gamma}$ is larger closer to the valley of the stability and
smaller closer to the neutron drip line (\citealt{iliadis2008}) and
so, for a given isotopic chain, if $Q_{n\gamma}$ is 3 MeV somewhere
between the valley of stability and the neutron drip line then closer
to the valley of stability $Q_{n\gamma}> 3$  MeV and $N(Z,A+1) >
N(Z,A)$; similarly, closer to the neutron drip line $Q_{n\gamma}< 3$
MeV and $N(Z,A+1) < N(Z,A)$.  Thus in this simple picture (no spin or
partition function considerations and the assumption  that neutron capture and
photodisintegration proceed faster than \bminus\ decays)
the r-process  tries to pull abundances in a
given isotopic chain toward a specific $^A_Z$X which has a
$Q_{n\gamma}$ value close to that which permits $N(Z,A+1) \approx N(Z,A)$. 

%An important extension to this model to make it more physically
%palatable is that, for isotopes adjacent in $A$, isotopes with even
%numbers of neutrons have smaller $Q_{n\gamma}$ values than those with
%odd numbers of neutrons.  

\bminus\ decays move nuclides from one isotopic chain to the next
higher one in $Z$.  Looking at elements as a whole \cite{iliadis2008}
has a simple model showing the movement of nuclides from one element
to the next.  Defining $N_Z$ to be the abundance of a given {\it
element}, not nuclide
\begin{equation}
\frac{dN_Z}{dt} = - \lambda_ZN_Z + \lambda_{Z-1}N_{Z-1}
\end{equation}

 Evidence that
something like the r-process is at work in the universe can be seen by
the abundance peaks in {\bf Figure r-process peak offset}.  As
mentioned in Section~\ref{sec:s} 
the lower neutron capture cross-section of nuclides with a magic
number of neutrons there is a pile up of the abundances of these
nuclides.   Because the r-process moves on the neutron-rich side
the valley of stability the nuclei with a magic number of neutrons
have less protons than the corresponding nuclei of the s-process;
thus, in this case the r-processed nuclei will be at a lower $A$ than the
s-processed nuclei both before and after \bminus\ decaying to the
valley of stability and the corresponding abundance peak of the
r-processed nuclei will be at lower $A$ than the s-process peak.  {\bf
Figure shows this}.  Additionally, because the number of neutron magic
number nuclei accessible to the r-process is greater than the
corresponding nuclei for the s-process, the r-process peaks are wider
than the s-process peaks.  The existence of these peaks is one of the
strongest evidences for the existence of something like the r-process.

Another strong evidence for something like the r-process being in
operation in the universe is the natural existence of Th and Ur.
These elements cannot be created by the s-process because of the
s-process termination described by Equations~\ref{eq:firsttermination}
and~\ref{eq:secondtermination}; their creation requires a process that
can bridge over nuclides with short decay times.

As was mentioned in Section~\ref{sec:s} there are some stable nuclides
that are inaccessible by the s-process due to ``weak links'' in the
reaction chains that would otherwise contribute to their formation,
the weak links being nuclides with fast decay rates that \bminus\
decay back to the valley of stability; the formation of these nuclides
is completely attributable to the r-process.  Similarly, there are
some nuclides that are inaccessible by the r-process and whose
formation can be completely attributed to the s-process.  {\bf Figure}
shows such a nuclide.  r-process reation chains would end up at ?? and
thus would be unable to access any isobars of ?? with higher $Z$.  ??
is shielded from the r-process by ?? and thus cannot be formed by the
s-process.



Waiting time approximation (also called primary r-process: \citealt{meyer1994}).

Dynamic r-process calculatyion (Nn, T descreatese with time


\subsection{Proposed Sites}

Entropy discussion of Meyer?

Proposed r-processing sites mainly include core-collapse supernovae
and neutron star mergers.  is greater than for
the s-process discussed in Section~\ref{sec:s} and also less certain.  

An important clue in determining a site or sites where the r-process
occurs is found in observations of old stars with very low
metallicity.   

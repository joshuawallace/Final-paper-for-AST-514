\label{sec:r}
\subsection{Description}

The r-process occurs when neutron
densities and energies are high enough that the timescale for
neutron capture is much shorter than for $\beta^-$ decay.  Nuclear
evolution in this case does not follow the valley of stability but
instead passes on the neutron-rich side of the valley of stability.
Eventually as the neutron capture rate decreases the nuclei will have
an opportunity to \bminus\ decay back to the valley of stability.

One piece of physics very relevant to this process is the so-called
``neutron drip line.''  A nucleus with a given $Z$ can only have so
many neutrons added before it starts to become energetically favorable
for neutrons to leave the nucleus.  When this number of neutrons is
reached the addition of another neutron causes a
neutron to promptly be emitted (or ``dripped'') from the nucleus.
This maximum number of neutrons a nucleus can hold expressed as a
function of $Z$ describes the neutron drip line and it is the hard
edge seen on the neutron-rich side of the CON.

Because the r-process does not follow the valley of stability like the
s-process does, it is impossible to describe a canonical r-process
trajectory as was done for the s-process.  Instead a general picture
will be described illustrating the general principles that determine
each of a multitude of r-process trajectories.  The r-process begins 
with heavy seed nuclei, such as $^{56}$Fe.  The
rapid succession of neutron captures on these nuclei cause a
signficant neutron enrichment of the nuclei; the results of this
neutron enrichment depend on many factors such as neutron density,
temperature, and the decay times of the created nuclides, but
generally one of several things can happen while the neutron source is
active: neutron capture proceeds
faster than any \bminus\ decays, producing neutron-rich species that
run up against the neutron drip line; \bminus\ decays can occur at a
rate comparable to the neutron capture rate and nuclei can move up in
$Z$ and $A$ on a neutron-rich path that roughly parallels the valley 
of stability; nuclei capture neutrons until their neutron number 
runs up against a nuclear magic number; the low neutron capture
cross-section can effectively shut down further neutron enrichment of
the nuclei; or a nucleus with a magic number of neutrons captures a
neutron and moves to another region of the CON where any of the above
scenarios can happen again.

After the neutron source is turned off the neutron-rich nuclei are
able to \bminus\ decay to the valley of stability.  Evidence that
something like the r-process is at work in the universe can be seen by
the abundance peaks in {\bf Figure r-process peak offset}.  As
mentioned in Section~\ref{sec:s} 
the lower neutron capture cross-section of nuclides with a magic
number of neutrons there is a pile up of the abundances of these
nuclides.   Because the r-process moves on the neutron-rich side
the valley of stability the nuclei with a magic number of neutrons
have less protons than the corresponding nuclei of the s-process;
thus, in this case the r-processed nuclei will be at a lower $A$ than the
s-processed nuclei both before and after \bminus\ decaying to the
valley of stability and the corresponding abundance peak of the
r-processed nuclei will be at lower $A$ than the s-process peak.  {\bf
Figure shows this}.  Additionally, because the number of neutron magic
number nuclei accessible to the r-process is greater than the
corresponding nuclei for the s-process, the r-process peaks are wider
than the s-process peaks.  The existence of these peaks is one of the
strongest evidences for the existence of something like the r-process.

Another strong evidence for something like the r-process being in
operation in the universe is the natural existence of Th and Ur.
These elements cannot be created by the s-process because of the
s-process termination described by Equations~\ref{eq:firsttermination}
and~\ref{eq:secondtermination}; their creation requires a process that
can bridge over nuclides with short decay times.

As was mentioned in Section~\ref{sec:s} there are some stable nuclides
that are inaccessible by the s-process due to ``weak links'' in the
reaction chains that would otherwise contribute to their formation,
the weak links being nuclides with fast decay rates that \bminus\
decay back to the valley of stability; the formation of these nuclides
is completely attributable to the r-process.  Similarly, there are
some nuclides that are inaccessible by the r-process and whose
formation can be completely attributed to the s-process.  {\bf Figure}
shows such a nuclide.  r-process reation chains would end up at ?? and
thus would be unable to access any isobars of ?? with higher $Z$.  ??
is shielded from the r-process by ?? and thus cannot be formed by the
s-process.


\subsection{Proposed Sites}

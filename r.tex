\label{sec:r}
\subsection{Description}

The r-process occurs when neutron
densities and energies are high enough that the timescale for
neutron capture is much shorter than for $\beta^-$ decay.  Nuclear
evolution in this case does not follow the valley of stability but
instead passes on the neutron-rich side of the valley of stability.
Eventually as the neutron capture rate decreases the nuclei will have
an opportunity to \bminus\ decay back to the valley of stability.

Several pieces of relevant physics must be introduced.  One, nuclear
magic numbers, was already covered in Section~\ref{sec:nuc}.  The
lower neutron capture cross-sections of nuclei with magic numbers of
neutrons has a significant impact on the flow of nuclei toward
increasing $A$ in the r process.  Another important piece 
 is the so-called
``neutron drip line.''  A nucleus with a given $Z$ has a maximum
number of neutrons it can keep bound to it; any additional neutrons
that are captured once this number is reached causes a neutron to be
emitted (or ``dripped'') from the nucleus.  So neutron capture at 
the neutron drip line permits no change in $A$ or $Z$.  
This maximum number of neutrons a nucleus can hold expressed as a
function of $Z$ describes the neutron drip line and it is the hard
edge seen on the neutron-rich side of the CON.  Also relevant is
photodisintegration.  At the temperatures expected for the r-process
photons have enough energy to liberate neutrons from the nucleus:
$^A_Z$X$+\gamma \rightarrow _{\ \ \ Z}^{A-1}$X$+n$.  This process
works against neutron capture.

A qualitative description of the r-process follows.  Because the 
r-process does not follow the valley of stability like the
s-process does, it is more difficult to describe a canonical r-process
trajectory than for the s-process.  Instead a general picture
will be described illustrating the general principles that determine
each of a multitude of r-process trajectories.  A
rapid succession of neutron captures on seed nuclei causes a
signficant neutron enrichment of the nuclei.  As this proceeds there
are several things that can happen.  Additional neutron captures can
further increase the $A$ of the nuclei.  Photodisintegration can
remove neutrons from the nucleus.

%; the results of this
%neutron enrichment depend on many factors such as neutron density,
%temperature, and the decay times of the created nuclides, but
%generally one of several things can happen while the neutron source is
%active: 

neutron captures continue faster than \bminus\ decays and the
nuclides continue to be enriched in neutrons

 neutron capture proceeds
faster than any \bminus\ decays, producing neutron-rich species that
run up against the neutron drip line; \bminus\ decays can occur at a
rate comparable to the neutron capture rate and nuclei can move up in
$Z$ and $A$ on a neutron-rich path that roughly parallels the valley 
of stability; nuclei capture neutrons until their neutron number 
runs up against a nuclear magic number; the low neutron capture
cross-section can effectively shut down further neutron enrichment of
the nuclei; or a nucleus with a magic number of neutrons captures a
neutron and moves to another region of the CON where any of the above
scenarios can happen again.

After the neutron source is turned off the neutron-rich nuclei are
able to \bminus\ decay to the valley of stability.  Evidence that
something like the r-process is at work in the universe can be seen by
the abundance peaks in {\bf Figure r-process peak offset}.  As
mentioned in Section~\ref{sec:s} 
the lower neutron capture cross-section of nuclides with a magic
number of neutrons there is a pile up of the abundances of these
nuclides.   Because the r-process moves on the neutron-rich side
the valley of stability the nuclei with a magic number of neutrons
have less protons than the corresponding nuclei of the s-process;
thus, in this case the r-processed nuclei will be at a lower $A$ than the
s-processed nuclei both before and after \bminus\ decaying to the
valley of stability and the corresponding abundance peak of the
r-processed nuclei will be at lower $A$ than the s-process peak.  {\bf
Figure shows this}.  Additionally, because the number of neutron magic
number nuclei accessible to the r-process is greater than the
corresponding nuclei for the s-process, the r-process peaks are wider
than the s-process peaks.  The existence of these peaks is one of the
strongest evidences for the existence of something like the r-process.

Another strong evidence for something like the r-process being in
operation in the universe is the natural existence of Th and Ur.
These elements cannot be created by the s-process because of the
s-process termination described by Equations~\ref{eq:firsttermination}
and~\ref{eq:secondtermination}; their creation requires a process that
can bridge over nuclides with short decay times.

As was mentioned in Section~\ref{sec:s} there are some stable nuclides
that are inaccessible by the s-process due to ``weak links'' in the
reaction chains that would otherwise contribute to their formation,
the weak links being nuclides with fast decay rates that \bminus\
decay back to the valley of stability; the formation of these nuclides
is completely attributable to the r-process.  Similarly, there are
some nuclides that are inaccessible by the r-process and whose
formation can be completely attributed to the s-process.  {\bf Figure}
shows such a nuclide.  r-process reation chains would end up at ?? and
thus would be unable to access any isobars of ?? with higher $Z$.  ??
is shielded from the r-process by ?? and thus cannot be formed by the
s-process.

That was a qualitative description of the r-process.  A quantitative
model for the r-process (termed the {\it classical} model
in \citealt{iliadis2008}) is given by
\begin{multline}
\frac{N(Z,A+1)}{N(Z,A)} = - N_nN(Z,A) \langle \sigma v \rangle_{Z,A} \\
+N(Z,A+1)\times\lambda_\gamma (Z,A+1)
\end{multline}
where $N(Z,A)$ is the number density of  nuclide $^A_Z$X, $\rangle 
\sigma v \langle_{Z,A}$ is the neutron
capture reaction rate per particle pair for the same nuclide, and
$\lambda_\gamma (Z,A+1)$ is the photodisintegration constant for
nuclide  $^{A+1}_{Z}$X.  But this doesn't include beta decays.

Waiting time approximation (also called primary r-process: \citealt{meyer1994}).

Dynamic r-process calculatyion (Nn, T descreatese with time


\subsection{Proposed Sites}

Entropy discussion of Meyer?

Proposed r-processing sites mainly include core-collapse supernovae
and neutron star mergers.  is greater than for
the s-process discussed in Section~\ref{sec:s} and also less certain.  

An important clue in determining a site or sites where the r-process
occurs is found in observations of old stars with very low
metallicity.   

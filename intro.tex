The creation of the variety of nuclides observed both terrestrially and throughout the universe is a topic not just of astrophysical interest but also important for understanding our origin as living organisms.  The relationship between our existence as living creatures and nucleosynthesis is often colloquially summed up as ``we are made from stardust''.  While this appears to be technically true (Big Bang nucleosynthesis aside), extant and developing research in nucleosynthesis show that the full story of how the nuclides came to be is far more complicated than a five word quip.  There are a variety of proposed sites and processes that all contribute to the measured elemental and isotopic abundances of the Galaxy which do not yet form a completely consistent picture.  This makes astrophysical nucleosynthesis  an active topic of research.  Section 2 covers a brief history of the development of theory behind the r- and s-process, while Section 3 covers the relevant nuclear physics.  The s-process is discussed in Section 4 and the r-process is discussed in Section 5.  Finally, brief conclusions are made in Section 6.
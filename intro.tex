The creation of the variety of elements and isotopes observed both terrestrially and throughout the universe is a topic not just of astrophysical importance but also important for understanding our origin as living organisms.  The relationship between our existence as living creatures and nucleosynthesis is colloquially summed up as ``we are made from stardust''.  While this is technically true (Big Bang nucleosynthesis aside), extant and developing research in nucleosynthesis show that the full story of how the elements and isotopes came to be is far more complicated than a five word quip.  There are a variety of proposed sites and processes that all contribute to the measured elemental and isotopic abundances of the Galaxy which, so far, still do not form a completely consistent picture.  This makes astrophysical nucleosynthesis still an active topic of research.

The field of nucleosynthesis was born...(history lesson)
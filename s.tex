\label{sec:s}
\subsection{Description}
The s-process occurs when
the rate for neutron capture is much lower than the rate
for \bminus\ decay.  Since rates for \bminus\ decay vary greatly
it is difficult to define a specific \bminus\ decay half-life as ``the
reference'' for specifying an upper-limit to the s-process neutron
capture rate.  Indeed, even in typical
s-process trajectories there are radioactive isotopes with
sufficiently long half-lives to cause branching canonical s-process
trajectory (which will be defined later) but, in practice, such
branching is just a small variation on an overall theme.  

A general trajectory of the s-process is shown in {\bf Figure}.  A
canonical trajectory of the s-process begins with a specific seed
nucleus (in physical scenarios and calculation $^{56}$Fe is often the
seed nucleus) which undergoes neutron capture (given by
Equation~\ref{eq:nc}) until an unstable isotope is reached, at which
point the nuclide will undergo \bminus\ decay (given by
Equation~\ref{eq:bd}) until another stable nuclide is reached, at
which point the nuclide will undergo neutron capture until an unstable
isotope is reached and the process repeats.  The key for this
canonical s-process trajectory is that once an unstable species is
reached no neutron captures happen before the species \bminus\
decays.  In reality, as is shown in {\bf Figure of branching} there
are nuclides with sufficiently long half-lives that neutron capture on
them is possible before they decay, thus producing a branch in the
s-process trajectory.  Also, because of differences between
environments where the s-process may occur the number of branches and
strength of each branch may also vary from environment to
environment.  Due to the narrowness of the valley of stability
branches typically converge after a few reactions.

Since the decay rate of radioactive isotopes is higher than the
neutron capture rate s-process trajectories through the chart of the
nuclides remain very close to the valley of stability.  This also
means that there are some stable nuclei which cannot be formed by the
s-process.  If a nuclide with a given atomic number $Z_1$ and atomic
mass $A_1$ has an isotope with atomic number $Z_1$ and atomic mass
$A_X$ (where $A_X \geq A_1 +2$) the s-process can only produce the
isotope with atomic mass $A_X$ if all the intervening isotopes with
atomic mass $A_Y$, $A_1 < A_Y < A_X$ are stable as well.  If the
nuclide with atomic mass $A_X$ cannot be reached through neutron
capture before the unstable isotope(s) with atomic mass $A_Y$ decay
then the s-process will never produce it.

The s-process has a termination point, at which no heavier elements
can be made.  One such termination loop beginning with $^{209}$Bi
(\citealt{claytonetal1961}):
\begin{equation*}
^{209}_{\ 83}\textrm{Bi} + n  \rightarrow ^{210}_{\ 83}\textrm{Bi} +
\gamma 
\end{equation*}
\begin{equation*}
^{210}_{\ 83}\textrm{Bi}   \rightarrow ^{210}_{\ 84}\textrm{Po} +
e^- + \overline{\nu}_e 
\end{equation*}
\begin{equation*}
^{210}_{\ 84}\textrm{Po}   \rightarrow ^{206}_{\ 82}\textrm{Pb} +
\alpha 
\end{equation*}
\begin{equation*}
^{206}_{\ 82}\textrm{Pb} + 3n  \rightarrow ^{209}_{\ 82}\textrm{Pb} +
3\gamma 
\end{equation*}
\begin{equation}
^{209}_{\ 82}\textrm{Pb}  \rightarrow ^{209}_{\ 83}\textrm{Bi} + e^-
+ \overline{\nu}_e
\end{equation}

which then loops.  \cite{claytonetal1961} say that the first step in this loop 56\% of
the time leads to the unstable $^{210}_{\ 83}\textrm{Bi}$ that decays
to $^{210}_{\ 84}\textrm{Po}$ while 44\% of the time is leads to a
nearly stable ground state of $^{210}_{\ 83}\textrm{Bi}$ with a
half-life of $2.6 \times 10^6$ years.  A neutron capture
on $^{210}_{\ 83}\textrm{Bi}$ proceeds as follows:
\begin{equation*}
^{210}_{\ 83}\textrm{Bi} + n  \rightarrow ^{211}_{\ 83}\textrm{Bi} +
\gamma
\end{equation*}
\begin{equation*}
^{211}_{\ 83}\textrm{Bi}  \rightarrow ^{207}_{\ 81}\textrm{Tl} + \alpha
\end{equation*}
\begin{equation*}
 ^{207}_{\ 81}\textrm{Tl}  \rightarrow  ^{207}_{\ 82}\textrm{Pb} + e^-
 + \overline{\nu}_e
\end{equation*}
\begin{equation*}
^{207}_{\ 82}\textrm{Pb} + 2n  \rightarrow ^{209}_{\ 82}\textrm{Pb} +
2\gamma
\end{equation*}
\begin{equation}
^{209}_{\ 82}\textrm{Pb}  \rightarrow ^{209}_{\ 83}\textrm{Bi} + e^- + \overline{\nu}_e
\end{equation}
which also loops.


\subsection{Proposed Sites}

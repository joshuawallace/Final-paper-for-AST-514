\label{sec:s}
\subsection{Description}
The s-process occurs when
the rate for neutron capture is much lower than the rate
for \bminus\ decay.  Since rates for \bminus\ decay vary greatly
it is difficult to define a specific \bminus\ decay half-life as ``the
reference'' for specifying an upper-limit to the s-process neutron
capture rate.  Indeed, even in typical
s-process trajectories there are radioactive isotopes with
sufficiently long half-lives to cause branching canonical s-process
trajectory (which will be defined later) but, in practice, such
branching is just a small variation on an overall theme.  

A general trajectory of the s-process is shown in {\bf Figure}.  A
canonical trajectory of the s-process begins with a specific seed
nucleus (in physical scenarios and calculation $^{56}$Fe is often the
seed nucleus) which undergoes neutron capture (given by
Equation~\ref{eq:nc}) until an unstable isotope is reached, at which
point the nuclide will undergo \bminus\ decay (given by
Equation~\ref{eq:bd}) until another stable nuclide is reached, at
which point the nuclide will undergo neutron capture until an unstable
isotope is reached and the process repeats.  The key for this
canonical s-process trajectory is that once an unstable species is
reached no neutron captures happen before the species \bminus\
decays.  In reality, as is shown in {\bf Figure of branching} there
are nuclides with sufficiently long half-lives that neutron capture on
them is possible before they decay, thus producing a branch in the
s-process trajectory.  Also, because of differences between
environments where the s-process may occur the number of branches and
strength of each branch may also vary from environment to
environment.  Due to the narrowness of the valley of stability
branches typically converge after a few reactions.

Since the decay rate of radioactive isotopes is higher than the
neutron capture rate s-process trajectories through the chart of the
nuclides remain very close to the valley of stability.  This also
means that there are some stable nuclei which cannot be formed by the
s-process.  If a nuclide with a given atomic number $Z_1$ and atomic
mass $A_1$ has an isotope with atomic number $Z_1$ and atomic mass
$A_X$ (where $A_X \geq A_1 +2$) the s-process can only produce the
isotope with atomic mass $A_X$ if all the intervening isotopes with
atomic mass $A_Y$, $A_1 < A_Y < A_X$ are stable as well.  If the
nuclide with atomic mass $A_X$ cannot be reached through neutron
capture before the unstable isotope(s) with atomic mass $A_Y$ decay
then the s-process will never produce it.

The s-process has a termination point, at which no heavier elements
can be made.  One such termination loop beginning with $^{209}$Bi
(\citealt{claytonetal1961}):
\begin{equation*}
^{209}_{\ 83}\textrm{Bi} + n  \rightarrow ^{210}_{\ 83}\textrm{Bi} +
\gamma 
\end{equation*}
\begin{equation*}
^{210}_{\ 83}\textrm{Bi}   \rightarrow ^{210}_{\ 84}\textrm{Po} +
e^- + \overline{\nu}_e 
\end{equation*}
\begin{equation*}
^{210}_{\ 84}\textrm{Po}   \rightarrow ^{206}_{\ 82}\textrm{Pb} +
\alpha 
\end{equation*}
\begin{equation*}
^{206}_{\ 82}\textrm{Pb} + 3n  \rightarrow ^{209}_{\ 82}\textrm{Pb} +
3\gamma 
\end{equation*}
\begin{equation}
\label{eq:firsttermination}
^{209}_{\ 82}\textrm{Pb}  \rightarrow ^{209}_{\ 83}\textrm{Bi} + e^-
+ \overline{\nu}_e
\end{equation}

which then loops.  \cite{claytonetal1961} say that the first step in this loop 56\% of
the time leads to the unstable $^{210}_{\ 83}\textrm{Bi}$ that decays
to $^{210}_{\ 84}\textrm{Po}$ while 44\% of the time is leads to a
nearly stable ground state of $^{210}_{\ 83}\textrm{Bi}$ with a
half-life of $2.6 \times 10^6$ years.  A neutron capture
on $^{210}_{\ 83}\textrm{Bi}$ proceeds as follows:
\begin{equation*}
^{210}_{\ 83}\textrm{Bi} + n  \rightarrow ^{211}_{\ 83}\textrm{Bi} +
\gamma
\end{equation*}
\begin{equation*}
^{211}_{\ 83}\textrm{Bi}  \rightarrow ^{207}_{\ 81}\textrm{Tl} + \alpha
\end{equation*}
\begin{equation*}
 ^{207}_{\ 81}\textrm{Tl}  \rightarrow  ^{207}_{\ 82}\textrm{Pb} + e^-
 + \overline{\nu}_e
\end{equation*}
\begin{equation*}
^{207}_{\ 82}\textrm{Pb} + 2n  \rightarrow ^{209}_{\ 82}\textrm{Pb} +
2\gamma
\end{equation*}
\begin{equation}
\label{eq:secondtermination}
^{209}_{\ 82}\textrm{Pb}  \rightarrow ^{209}_{\ 83}\textrm{Bi} + e^- + \overline{\nu}_e
\end{equation}
which also loops.

Because nuclei with magic numbers of neutrons have a lower neutron
capture cross-section there is a pile up of these nuclei relative to
their neighbors on the CON.  The s-process peaks can be seen in {\bf
Figure s process peaks}. Because the s-process follows the valley of
stability so closely there are only a few nuclei created by the
s-process with a magic number of neutrons, which leads to the narrow
peaks seen in the {\bf Figure}.  This is to be contrasted with the
relatively broader peaks of the r-process as will be described in
Section~\ref{sec:r}. 

For the canonical s-process, which assumes that the decay time of
unstable nuclides is much smaller than the timescale for neutron
capture in all radioactive nuclei, there is only one stable nucleus
for a given $A$ accessible by the s-process and thus the s-process
path is uniquely defined by $A$.  Under these assumptions the abudance
evolution of a stable nuclide is given by (\citealt{iliadis2008})
\begin{multline}
\label{eq:sreactionrate}
\frac{dN_s(A,t)}{dt} = - N_n(t)N_s(A,t)\langle \sigma v \rangle_A \\
+N_n(t)N_s(A-1,t)\langle \sigma v \rangle_{A-1}
\end{multline}
where $N_s(A)$ is the number density of nucleus $A$, $N_n(t)$ is the
number density of neutrons which (in general) can depend on time, and
$\langle \sigma v \rangle_A$ is the
neutron capture reaction rate per particle pair of nucleus $A$.
Equation~\ref{eq:sreactionrate} is instructive in its simplicity; the
abundance of any nuclide on a canonical s-process track is determined
entirely by the number density of free neutrons, the nuclide's neutron
capture cross-section, the neutron capture cross-section for the
nuclide previous to it on the s-process track, and the time for which
the s-process is active.  This also shows that, if there is only a
limited amount of seed nuclei to start this process, the s-process
will constantly evolve in time toward heavier nuclei with the eventual
result being all s-processed nuclei becoming the terminating isotopes
of Bi and Pb.  

A more general trajectory which allows for the possibility of
branching is given by
\begin{multline}
\frac{dN_s(Z,A)}{dt}
\end{multline}
.

I assume that the temperature $T$ is constant over the time the
s-process is active.  Assuming a Maxwell-Boltzmann distribution 
for the neutron velocities, 
$\langle \sigma v \rangle_A$ can be written as
$ \overline{\sigma}_A v_T$, where $ \overline{\sigma}_A$ is the
Maxwellian averaged cross-section and $v_T$ is the thermal velocity of
the neutrons given by $v_T = \sqrt{2kT/m_N}$.  In reality the term
$m_N$ in the previous expression should be the reduced mass but for the
nuclei we will be looking at ($A \gtrsim 60$) approximating the
reduced mass as $m_N$ is reasonable enough for the calculations shown
here.  It is also useful to define the neutron exposure $\tau$ as
\begin{equation}
\tau = v_T \int N_n(t)dt
\end{equation}
which will more useful expressed in differential form
\begin{equation}
\label{eq:dtau}
d\tau = v_TN_n(t)dt.
\end{equation}
The neutron exposure as defined here has units of neutrons per area
and is comparable to the neutron fluence.  Substituting
Equation~\ref{eq:dtau} into Equation~\ref{eq:sreactionrate} gives
\begin{equation}
\label{eq:intermsoftau}
\frac{dN_s(A,\tau)}{d\tau} = -N_s(A,\tau)\overline{\sigma}_A +
N_s(A-1,\tau)\overline{\sigma}_{A-1}.
\end{equation}
This allows us to quantitatively determine whether a nuclide's
abundance will increase or decrease,
\begin{equation*}
\frac{dN_s(A,\tau)}{d\tau} < 0 ~~~\textrm{if}~~~ N_s(A,\tau)
> \frac{\overline{\sigma}_{A-1}}{\overline{\sigma}_{A}}N_s(A-1,\tau) 
\end{equation*}
\begin{equation
\frac{dN_s(A,\tau)}{d\tau} > 0 ~~~\textrm{if}~~~ N_s(A,\tau)
< \frac{\overline{\sigma}_{A-1}}{\overline{\sigma}_{A}}N_s(A-1,\tau).
\end{equation}
Thus, the abundance of a nuclide $A$ in the canoncial s-process with
constant $T$ depends only on its neutron capture cross-section, the
neutron capture cross-section and abundance of the species before it 
in the process, and the neutron exposure $\tau$, a measure of neutron
fluence.

\cite{iliadis2008} notes that the recursively coupled equations given
by Equation~\ref{eq:intermsoftau} are self-regulating, meaning that
they try to minimize the difference
$N_s(A-1,\tau)\overline{\sigma}_{A-1} -
N_s(A,\tau)\overline{\sigma}_{A}$.  
%You can continue with this thought if you want.


\subsection{Proposed Sites}

\subsection{Description}
The s-process ({\it s} means {\it slow neutron capture}) occurs when
the rate for neutron capture is much lower than the rate
for \bminus\ decay.  Since rates for $bminus$ decay vary greatly
it is difficult to define a specific $bminus$ decay half-life as ``the
reference'' for specifying an upper-limit to the s-process neutron
capture rate.  Indeed, even in typical
s-process trajectories there are radioactive isotopes with
sufficiently long half-lives to disrupt a canonical s-process
trajectory (which will be defined later) but, in practice, such
disruptions are just small variations on an overall theme.  

Since the decay rate of radioactive isotopes is higher than the
neutron capture rate s-process trajectories through the chart of the
nuclides remain very close to the valley of stability.


\subsection{Proposed Sites}
